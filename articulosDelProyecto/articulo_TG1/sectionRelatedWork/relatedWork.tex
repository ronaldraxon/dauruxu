\documentclass[]{article}
\usepackage{lmodern}
\usepackage{amssymb,amsmath}
\usepackage{ifxetex,ifluatex}
\usepackage{fixltx2e} % provides \textsubscript
\ifnum 0\ifxetex 1\fi\ifluatex 1\fi=0 % if pdftex
  \usepackage[T1]{fontenc}
  \usepackage[utf8]{inputenc}
\else % if luatex or xelatex
  \ifxetex
    \usepackage{mathspec}
  \else
    \usepackage{fontspec}
  \fi
  \defaultfontfeatures{Ligatures=TeX,Scale=MatchLowercase}
\fi
% use upquote if available, for straight quotes in verbatim environments
\IfFileExists{upquote.sty}{\usepackage{upquote}}{}
% use microtype if available
\IfFileExists{microtype.sty}{%
\usepackage{microtype}
\UseMicrotypeSet[protrusion]{basicmath} % disable protrusion for tt fonts
}{}
\usepackage[margin=1in]{geometry}
\usepackage{hyperref}
\hypersetup{unicode=true,
            pdftitle={Motivacion},
            pdfauthor={Ronald},
            pdfborder={0 0 0},
            breaklinks=true}
\urlstyle{same}  % don't use monospace font for urls
\usepackage{graphicx,grffile}
\makeatletter
\def\maxwidth{\ifdim\Gin@nat@width>\linewidth\linewidth\else\Gin@nat@width\fi}
\def\maxheight{\ifdim\Gin@nat@height>\textheight\textheight\else\Gin@nat@height\fi}
\makeatother
% Scale images if necessary, so that they will not overflow the page
% margins by default, and it is still possible to overwrite the defaults
% using explicit options in \includegraphics[width, height, ...]{}
\setkeys{Gin}{width=\maxwidth,height=\maxheight,keepaspectratio}
\IfFileExists{parskip.sty}{%
\usepackage{parskip}
}{% else
\setlength{\parindent}{0pt}
\setlength{\parskip}{6pt plus 2pt minus 1pt}
}
\setlength{\emergencystretch}{3em}  % prevent overfull lines
\providecommand{\tightlist}{%
  \setlength{\itemsep}{0pt}\setlength{\parskip}{0pt}}
\setcounter{secnumdepth}{0}
% Redefines (sub)paragraphs to behave more like sections
\ifx\paragraph\undefined\else
\let\oldparagraph\paragraph
\renewcommand{\paragraph}[1]{\oldparagraph{#1}\mbox{}}
\fi
\ifx\subparagraph\undefined\else
\let\oldsubparagraph\subparagraph
\renewcommand{\subparagraph}[1]{\oldsubparagraph{#1}\mbox{}}
\fi

%%% Use protect on footnotes to avoid problems with footnotes in titles
\let\rmarkdownfootnote\footnote%
\def\footnote{\protect\rmarkdownfootnote}

%%% Change title format to be more compact
\usepackage{titling}

% Create subtitle command for use in maketitle
\providecommand{\subtitle}[1]{
  \posttitle{
    \begin{center}\large#1\end{center}
    }
}

\setlength{\droptitle}{-2em}

  \title{Motivacion}
    \pretitle{\vspace{\droptitle}\centering\huge}
  \posttitle{\par}
    \author{Ronald}
    \preauthor{\centering\large\emph}
  \postauthor{\par}
    \date{}
    \predate{}\postdate{}
  

\begin{document}
\maketitle

\hypertarget{related-work}{%
\section{Related work}\label{related-work}}

Within the context of psychosocial risk factors, some variants may be
inherent individually or together in a work environment. It is essential
to clarify that the environments can be external when working outdoors
and internal when working indoors. The most common types of risks for
both cases are: Physical risks (also known as workplace risk) refer to
aspects of the environment where the work takes place. Among the most
significant aspects are noise, lighting, or the temperature of the
environment{[}@MIRZA2018{]}{[}@NIELSEN2018{]}. Chemical risks are highly
related to industrial environments where any worker may have contact
with dust, gases, or abrasive products{[}@SHIN2014{]}{[}@NIJ2017{]}.
Biological risks involve contact with living things such as fungi,
bacteria, or viruses, particularly by the interaction with people who
may have a disease, infections, animals, or plants that may be carriers
of a harmful organism{[}@CORRAO2012{]}{[}@MORIKAWA2012{]}. Mechanical
risks may be associated with some aspects of the work environment. It is
related to heavy machinery usage or the development of an activity in
which any person exposes to the effects of vibration
{[}@PALMER2003{]}{[}@SUNDSTRUP2017{]}. Environmental type risks involve
scenes or work, where there is a high probability of floods, storms, or
contamination{[}@MARSHALL2016{]}{[}@ANTHONJ201934{]}. Finally,
psychosocial risks occur in the normal execution of daily activities.
These are strongly related to the work conditions, people's interaction,
and socio-demographic conditions. Among the most studied aspects, it is
stress, monotony, and job fatigue due to excess hours
worked{[}@ROCHA2014{]}{[}@RAFFO2014{]}. As this last type of risk is the
main focus of the present work, section 2.1 will present the evaluation
methods.

\hypertarget{psychosocial-risk-assesment-pria}{%
\subsection{Psychosocial Risk Assesment
(PRIA)}\label{psychosocial-risk-assesment-pria}}

Currently, some methods facilitate the evaluation of FRP developed from
the integration of models and scales, which seek to qualify risk
factors. In works such as Charria, Sarsosa, and Arenas{[}@CHARRIA2011{]}
is suggested a taxonomy of mechanisms, taking into account the form in
the information extracted and its scope. In this work, there are two
large groups of questionnaires oriented to industrial hygiene and
psychosocial factors. In the first group, evaluates aspects such as the
work environment, the physical effects on workers, and details of hiring
and remuneration. The assessments of these aspects use questionnaires
that are carried out by an external agent to the organization, who seeks
an objective evaluation of the situation. Some examples of this group
are the Questionnaire for the Fifth European Survey on Working
Conditions{[}{]} and the Quality of Life at Work Survey
Questionnaire{[}{]}. In the second group, there are questionnaires
oriented to psychosocial factors acquired through interviews or a
self-report procedure. Interview questionnaires collect information
related to job satisfaction, burnout, or bullying. On the other hand,
self-report questionnaires extract information related to individual
aspects of the person, such as the relationship between health and
illness, aspects of daily life, and their social interactions. Some
examples of this second group are the EAE Stress Assessment
questionnaire{[}{]}, the occupational burnout scale{[}{]}, the
Bocanument and Berján evaluation{[}{]}, and the Demand-Control
model{[}{]}.

Concerning the groups of questionnaires mentioned, there are
investigations which reveals that some conditions generate effects
related to physical health such as musculoskeletal
disorders{[}@ANDERSON1997{]} or the behavior of people such as sedentary
lifestyle {[}@MORALES2014{]}. On the other hand, other studies show
effects related to people's mood{[}@RHEE2017{]} with mental health such
as stress {[}@AZUMA2015{]} and psychological disorders such as anxiety
{[}@WIEGNER2015{]} or depression {[}@LUCA2014{]} {[}@WINSOR2016{]}.
Although the psychosocial risk is widely related to work, it is not
exclusive to these environments. Researches such as that of Abdullah
Alotaibi {[}@ABDULLAH2020{]}, Christian Hederich{[}@HEDERICH2016{]}, and
Malarvili{[}@MALARVILI2018{]} address the relationship between sleep
quality and stress in academic settings. Within the research carried out
in the academic context, There are studies of the prevalence and
correlation of depression, anxiety, and suicidal tendencies such as
Eisenberg's {[}@EISENBERG2007{]}. Other approaches, such as
Danuta's{[}@ZARZYCKA2014{]}, seek to identify the relationship of
demographic aspects such as the students' place of residence as
intervening variables in their state of health. It is also essential to
show that in these scenarios, students are not the only actors prone to
risk factors. Works such as that of Briones{[}@BRIONES2010{]} and
Pedditzi{[}@PEDDITZI2014{]} show a presence of stress and job exhaustion
among teachers.

During the last years, many mechanisms have been developed in the form
of questionnaires. These mechanisms have favored the improvement of
interactions at work, the conditions of their organization, as well as
the worker's abilities, needs, culture, and the personal situation
outside of work, all of which, through perceptions and experiences, can
influence health and performance and job satisfaction. However, the
influence not only comes from the work environment{[}@IZQUIERDO2012{]}
but also from the extra-work environment{[}@JIN2017{]}. In this last
aspect, psychosocial assessment methods seek to evaluate aspects such as
time away from work activities, family relationships, the economy of the
family group, commuting to work, among others. Some derivations or
generalizations of the exposed evaluation methods have contributed to
the improvement of well-being and good practices in the academic
context, promoted or the development of a mechanism for the promise of
stress management evidenced in Collen's work{[}@COLLEN2013{]}. Other
contributions have allowed approaches to identify the behaviors
associated with happiness, well-being, and the stress perceived in
university students{[}@CALDERON2019{]}.

The diversity of scenarios where evaluation methods play an essential
role, in turn, entails a series of challenges of experimental
validation, in which the aim is to establish correlation values of the
aspects evaluated with the real scenario{[}@RUBIO-CASTRO2015{]} or its
factor structure{[}@BLANCH2010{]}. Although there is high statistical
support for several of the items raised within the questionnaires, it
can be evidenced that the mechanisms and procedures are susceptible to
variability and subjectivity in the
measures{[}@CAICOYA2004{]}{[}@RICK2000{]}. Experimentations have the
caveat that samples are related to a particular segment of the
population. Also, some items in questionnaires assess relevant aspects
of daily activities that are not observed by specialists in occupational
safety and health that a. This last issue reduces the amount of evidence
drastically to establish reference values{[}@BENAVIDES2002{]}.

\hypertarget{technologic-approaches-supporting-pria}{%
\subsection{Technologic approaches supporting
PRIA}\label{technologic-approaches-supporting-pria}}

Some references have addressed some aspects related to the mental health
of people in the workplace{[}@CHOI2018{]}{[}@GOLONKA2019{]}. Some of
these works have resulted in technological solutions for monitoring some
specific aspects of psychosocial risk, ranging from the implementation
of load controls on the extremities and other parts of the body based on
sensors{[}@HUANG2012{]}. Other approaches focus on reducing accidents by
detecting elements or obstacles that can generate an accident. Among
these approaches, works that identify liquid spills or tools oriented to
the environment can be noted{[}@SEO2012{]}. On the other hand, to
identify aspects related to the mental condition in people, approaches
have been made through the use of artificial intelligence and computer
vision. In some of these approaches, electroencephalogram images
analysis is used to assess stress in people{[}@JEBELLI2018{]}. Other
works such as those by Zack Zhu{[}@ZHU2016{]} or Raffaele
Gravina{[}@GRAVINA2019{]}, suggest alternative perspectives, based on
the recognition of mood, from the capture of signals with portable
electronic devices.

Other approaches address the capture and integration with other data
sources, resulting in multimodal
architectures{[}@MAGDIN2016{]}{[}@SOLEYMANI2017{]}, in which the
processing of video images, text, signals, among others, is used to
support the diagnosis of emotions{[}@HARLEY2015{]}. Works such as that
of Le Yang{[}@YANG2017{]} and Poria Soujana{[}@PORIA2017{]} suggest the
fusion of paralinguistic analysis, capturing interview responses,
features of the face widely addressed{[}@JAIN2018{]}{[}@ZHU2018{]}, and
eye movement{[}@ALGHOWINEM2018{]}. Some approaches are oriented to
detect the effects of psychosocial risk factors, such as stress by
performance demands{[}@DINGES2005{]} and depression.

\textbf{Lorem ipsum dolor sit amet, consectetur adipiscing elit, sed
eiusmod tempor incidunt ut labore et dolore magna aliqua. Ut enim ad
minim veniam, quis nostrud exercitation ullamco laboris nisi ut aliquid
ex ea commodi consequat. Quis aute iure reprehenderit in voluptate velit
esse cillum dolore eu fugiat nulla pariatur. Excepteur sint obcaecat
cupiditat non proident, sunt in culpa qui officia deserunt mollit anim
id est laborum.Lorem ipsum dolor sit amet, consectetur adipiscing elit,
sed eiusmod tempor incidunt ut labore et dolore magna aliqua}. Although
these advances represent significant potential for the manufacturing
industry, construction, among others{[}@REID2017{]}, there are studies
such as that of Shall Mark{[}@SCHALL2018{]}, where there is evidence of
limitations for its adoption, the implications of cost; the interruption
of work activities, the intrusive nature represented in the discomfort
with the devices and the privacy of people.

In these approaches, a significant contribution is evident in the
analysis of voice patterns, and some aspects of interest are addressed
within the evaluation of FRP. However, the video mode used in the posts
above, focuses only on facial recognition, requiring the close-up
capture of people's faces and the use of sensors. Also, constant
monitoring considered.


\end{document}
